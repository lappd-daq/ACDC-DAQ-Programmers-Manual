
\subsection{Template}\label{command:template}
\subsubsection{Bit Fields}

\begin{figure}[hb!]
\begin{center}
\begin{bytefield}[endianness=big,bitwidth=1.1em]{32}
\bitheader{0-31} \\ 
\bitbox{8}{MSB} & \bitbox{4}{0xE} & \bitbox{20}{Data} \\
\end{bytefield}
\caption{Command 0xE bit fields}
\end{center}
\end{figure}

\begin{table}[h]
\begin{tabular}{p{0.15\textwidth}p{0.20\textwidth}p{0.65\textwidth}}
Bit Range & Name &  Description\\
\hline
31-24 & MSB & This is really the most significant byte \\
23-20 & \textbf{0xE} & Command marker \\
19-0 & Data & All of the data.\\
\hline
\end{tabular}
\end{table}



\subsubsection{Source}
\textbf{acdc-daq:src/DAQinstruction.cpp}
\begin{lstlisting}[style=c++]
void SuMo::software_trigger_slaveDevice(unsigned int SOFT_TRIG_MASK, bool set_bin, unsigned int bin)
{
    usb2.createHandles();
    //usb.sendData((unsigned int)0x000E0000);//software trigger
    const unsigned int hi_cmd = 0x000E0000;    
    unsigned int send_word = hi_cmd | SOFT_TRIG_MASK | set_bin << 4 | bin << 5;
    usb2.sendData(send_word);
    //printf("sent software trigger\n");
    usb2.freeHandles();
}
\end{lstlisting}



\subsubsection{Destination}
\textbf{ACDC-Firmware:SRC/dff\_async\_rst.vhd}  % project:path/to/filename.txt
\begin{lstlisting}[style=vhdl]
library IEEE;
use IEEE.STD_LOGIC_1164.all;
use IEEE.numeric_std.all;


entity dff_async_rst is
  port( d, clk, reset: in std_logic;
      q: out std_logic); 
end dff_async_rst;
architecture behav of dff_async_rst is
  begin
  process(clk,reset)
  begin
  if (reset='1') then 
    q <= '0';
  elsif rising_edge(clk) then 
    q <= d;
  end if;
  end process;
end behav;
\end{lstlisting}


