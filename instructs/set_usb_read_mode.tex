
\subsection{0xC - Set USB Read Mode}\label{inst:setusbreadmode}

This command sets the USB read mode. In the source code a few different modes are called frequently: 0, 5, and 16. 0 Corresponds to passing the USB instruction straight to the ACDC. 5 corresponds to reading out the CC buffer. 16 corresponds to setting the trig delay to 1.

\subsubsection{Bit Fields}
\begin{figure}[hb!]
	\begin{center}
		\begin{bytefield}[endianness=big,bitwidth=1.1em]{32}
			\bitheader{0-31} \\ 
			% Fill in bytefield here
			\bitbox{3}{\color{lightgray}\rule{\width}{\height}} & 
			\bitbox{4}{brd adr} &
			\bitbox{5}{\color{lightgray}\rule{\width}{\height}} & 
			\bitbox{4}{0xC} &
			\bitbox{1}{\color{lightgray}\rule{\width}{\height}} & 
			\bitbox{3}{src} & 
			\bitbox{7}{delay} & 
			\bitbox{1}{\rotatebox{90}{\small set}} & 
			\bitbox{1}{\rotatebox{90}{\scriptsize  trig}} &
			\bitbox{3}{mode}\\
		\end{bytefield}
		\caption{0x6} % Appropriate caption here
	\end{center}
\end{figure}

\begin{table}[h]
\begin{tabular}{p{0.06\textwidth}p{0.30\textwidth}p{0.64\textwidth}}
	Bits  & Name            & Description                                                      \\ \hline
	28-25 & board address   & The address of the board.                                        \\
	19-16 & \textbf{0xC}    & Command marker                                                   \\
	14-12 & trig source     & If 4 is enabled, defines the trig source                         \\
	11-5  & trig delay      &                                                                  \\
	4     & set trig source & If enabled, sets the trig source from bits 14-12                 \\
	3     & trig mode       &                                                                  \\
	2-0   & read mode       & The value of the read mode. Shown further in \autoref{readmodes} \\ \hline
\end{tabular}
\end{table}

\subsubsection{Read Modes}\label{readmodes}

\begin{table}[h]
	\begin{tabular}{p{0.06\textwidth}p{0.30\textwidth}p{0.64\textwidth}}
		Bits & Mode                     & Description                \\ \hline
		000  & ACDC instruction         &                            \\
		111  & ACDC trig validd CC only &                            \\
		101  & Read CC buffer           & The value of the read mode \\
		110  &                          &                            \\ \hline
	\end{tabular}
\end{table}


\subsubsection{Source}

\textbf{acdc-daq:src/DAQinstruction.cpp}
\begin{lstlisting}[style=c++]
void SuMo::set_usb_read_mode(unsigned int READ_MODE)
{
	usb.createHandles();
	const unsigned int hi_cmd = 0x000C0000;    
	unsigned int send_word = hi_cmd | READ_MODE | 15 << boardAdrOffset; 
	usb.sendData(send_word);
	usb.freeHandles();
}
\end{lstlisting}



\subsubsection{Destination}

\textbf{ACC-2xSPF Firmware:src/usbWrapperACC.vhd}  % project:path/to/filename.txt
\begin{lstlisting}[style=vhdl]
when x"C" =>

CC_READ_MODE <= USB_INSTRUCTION(2 downto 0);
if USB_INSTRUCTION(4) = '1' then
	TRIG_MODE <= USB_INSTRUCTION(3);
	TRIG_DELAY  (6 downto 0) <= USB_INSTRUCTION(11 downto 5);
	SET_TRIG_SOURCE (2 downto 0) <= USB_INSTRUCTION(14 downto 12);
end if;

if delay > 10 then
	delay := 0;
	state <= st1_WAIT;
--this is a hack:
-- basically, only want to send along to AC/DC if certain conditions apply
-- also want to only read CC info buffer if read mode = 0b101
else
	delay := delay + 1;
	if delay > 1 then
		case CC_READ_MODE is
		when "101" =>
			read_cc_buffer <= '1';
			CC_INSTRUCTION <= (others=>'0');
			CC_INSTRUCT_RDY<= '0';
		when "110" =>
			read_cc_buffer <= '0';
			CC_INSTRUCTION <= (others=>'0');
			CC_INSTRUCT_RDY<= '0';
		---only send-along data to AC/DC cards when 111 or 000
		when "111" =>	
			--cc_only_instruct_rdy <= '1';
			trig_valid_CC_only <= '1';
			read_cc_buffer <= '0';
			CC_INSTRUCTION <= (others=>'0');
			CC_INSTRUCT_RDY<= '0';
		when "000" =>
			trig_valid_CC_only <= '0';
			read_cc_buffer <= '0';
			CC_INSTRUCTION <= USB_INSTRUCTION;
			CC_INSTRUCT_RDY<= '1';
		------
		when others =>
			read_cc_buffer <= '0';
			CC_INSTRUCTION <= (others=>'0');
			CC_INSTRUCT_RDY<= '0';
		end case;
	end if;
end if;
\end{lstlisting}

