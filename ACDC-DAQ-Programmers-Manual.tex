\documentclass[11pt]{article}
\usepackage{geometry}        
\geometry{letterpaper}    
\usepackage[parfill]{parskip}  
\usepackage{amsmath, amssymb}
\usepackage{graphicx}
\usepackage{epstopdf}
\usepackage{titletoc}
\usepackage{tikz-timing}
\usepackage{bytefield}
\usepackage{listings}
\lstdefinelanguage{VHDL}{
   morekeywords={
     library,use,all,entity,is,port,in,out,end,architecture,of,
     begin,and
   },
   morecomment=[l]--
}
\usepackage{xcolor}
\colorlet{keyword}{blue!100!black!80}
\colorlet{comment}{green!90!black!90}
\lstdefinestyle{vhdl}{
   language     = VHDL,
   basicstyle   = \ttfamily,
   keywordstyle = \color{keyword}\bfseries,
   commentstyle = \color{comment},
   frame = single,
   breaklines=true,
   postbreak=\mbox{\textcolor{red}{$\hookrightarrow$}\space},
}
\lstdefinestyle{c++}{
   language = C++, 
   keywordstyle = \color{keyword}\bfseries,
   commentstyle = \color{comment},
   frame = single,
   breaklines=true,
   postbreak=\mbox{\textcolor{red}{$\hookrightarrow$}\space},
}
\usepackage[capitalise,nameinlink,noabbrev]{cleveref}

% Workaround for bytefield to make sure characters with tails appear on the same baseline
\newlength{\maxheight}
\setlength{\maxheight}{\heightof{W}}
\newcommand{\baselinealign}[1]{%
	\centering
	\raisebox{0pt}[\maxheight][0pt]{#1}%
}


% color for use in 'unused bytes' for bytefield
\definecolor{lightgray}{gray}{0.8}

\DeclareGraphicsRule{.tif}{png}{.png}{`convert #1 `dirname #1`/`basename #1 .tif`.png}

\usepackage[colorlinks=true, pdfstartview=FitV, linkcolor=blue, 
            citecolor=blue, urlcolor=blue]{hyperref}

%\includeonly{Chapter1}


% ------------------- Title and Author -----------------------------
\title{ACDC DAQ Programers Manual}
\author{Jonathan Eisch, Miles Lucas, Eric Oberla and Others} % Add your name here
\date{}
\begin{document}

\maketitle

\tableofcontents

\section{Introduction}\label{sec:intro}
This document is intended to document the commands sent to the ACC and ACDC cards.  

It includes source listings of the source and destinations for each command as they were when this documentation was written.  New implementations may have been written since, but that is outside the scope of this document.

The command protocol for each 32-bit instruction set is shown in \autoref{byte:protocol}.

\begin{figure}[hb!]
	\begin{center}
		\begin{bytefield}[boxformatting=\baselinealign, endianness=big,bitwidth=1.1em]{32}
			
			\bitheader{0-31} \\ 
			\bitbox{3}{open} & \bitbox{4}{FE mask} & \bitbox{5}{chip mask} & 
			\bitbox{4}{instruct} & \bitbox{4}{opt inst} & \bitbox{12}{value}\\
			
		\end{bytefield}
		\caption{Command protocol}
		\label{byte:protocol}
	\end{center}
\end{figure}

This is parsed on the front-end board like this \\ 

\begin{lstlisting}[style=vhdl]
INSTRUCT_PSEC_MASK	<= xINSTRUCT_WORD(24 downto 20); 
INSTRUCTION             <= xINSTRUCT_WORD(19 downto 16); 
INSTRUCTION_OPT		<= xINSTRUCT_WORD(15 downto 12); 
INSTRUCT_VALUE 		<= xINSTRUCT_WORD(11 downto 0); 
\end{lstlisting}

\autoref{tab:instructs} lists all the instruction flags passed by software.

\begin{table}[]
	\centering
	\caption{List of Instructions}
	\begin{tabular}{p{0.10\textwidth} p{0.6\textwidth}}
		Bits & Description                               \\ \hline
		0x0  & Do nothing                                \\
		0x1  & Set dealy-locked loop VDD control voltage \\
		0x2  & Calibration pulse switch enable           \\
		0x3  & Set pedestal                              \\
		0x4  & Reset DLL                                 \\
		0x5  & Reset internal trigger                    \\
		0x6  & Set self trigger mask                     \\
		0x7  & Set self trigger instructions             \\
		0x8  & Set trigger threshold                     \\
		0x9  & Adjust ring oscillation frequency         \\
		0xA  & Enable/disable on-board LEDs              \\
		0xB  & (DC) Central card FIFO toggle             \\
		0xB  & (CC) Central card done                    \\
		0xC  & (CC) Central card read mode               \\
		0xD  & (CC) Align/setup SERDES                   \\
		0xE  & (CC) USB trigger                          \\
		0xF  & (CC) Sync USB                             \\ \hline
	\end{tabular}
	\label{tab:instructs}
\end{table}

The instruction for setting self-trigger instructions has many optional instructions, listed in \autoref{tab:selftriginstructs}.

\begin{table}
	\centering
	\caption{Optional instructs for 0x7 - set self-trigger instructions}
	\begin{tabular}{p{0.10\textwidth} p{0.6\textwidth}}
		Bits & Description             \\ \hline
		0x0  & Enable                  \\
		0x1  & Wait for system trigger \\
		0x2  & Measure rate only       \\
		0x3  & Trigger sign            \\ \hline
	\end{tabular}
	\label{tab:selftriginstructs}
\end{table}
\startcontents[chapters]
\newpage


\section{Instructions}\label{sec:instructions}
\startcontents[chapters]
\printcontents[chapters]{}{1}{}

\subsection{Template}\label{inst:template}

% Add description of command here

\subsubsection{Bit Fields}
\begin{figure}[hb!]
\begin{center}
	\begin{bytefield}[endianness=big,bitwidth=1.1em]{32}
	\bitheader{0-31} \\ 
	% Fill in bytefield here
	\bitbox{32}{} \\ 
	\end{bytefield}
\caption{Command ... bit fields} % Appropriate caption here
\end{center}
\end{figure}

\begin{table}[h]	
% fill out bit information here
\begin{tabular}{p{0.06\textwidth}p{0.30\textwidth}p{0.64\textwidth}}
	Bits & Name & Description \\ \hline
	31-0 & Bits &             \\ \hline
\end{tabular}
\end{table}



\subsubsection{Source}
\textbf{project:path/to/filename.cpp}
% Put c++ source code here
\begin{lstlisting}[style=c++]

\end{lstlisting}



\subsubsection{Destination}
\textbf{project:path/to/filename.vhd}  % project:path/to/filename.txt
%put vhdl firmware code here
\begin{lstlisting}[style=vhdl]

\end{lstlisting}

\newpage

% add newly added commands here:
%\include{commands/newcommand}

\subsection{0xF - Sync Usb}\label{inst:syncusb}

This command does something regarding to syncing the usb

\subsubsection{Bit Fields}

\begin{figure}[hb!]
\begin{center}
\begin{bytefield}[endianness=big,bitwidth=1.1em]{32}

	\bitheader{0-31} \\ 
	\bitbox{12}{\color{lightgray}\rule{\width}{\height}} & \bitbox{4}{0xF} & 
	\bitbox{15}{\color{lightgray}\rule{\width}{\height}} \bitbox{1}{\rotatebox{90}{\small en}} \\

\end{bytefield}
\caption{Command 0xF bit fields}
\end{center}
\end{figure}

\begin{table}[h]
\begin{tabular}{p{0.06\textwidth}p{0.30\textwidth}p{0.64\textwidth}}
	Bits  & Name         & Description     \\ \hline
	19-16 & \textbf{0xF} & Command marker  \\
	0     & enable       & Enable USB sync \\ \hline
\end{tabular}
\end{table}



\subsubsection{Source}

\textbf{acdc-daq:src/DAQinstruction.cpp}
\begin{lstlisting}[style=c++]
void SuMo::sync_usb(bool SYNC) {
     createUSBHandles();
     if(SYNC != false) { //enable USB_SYNC
 	 usb.sendData((unsigned int)0x000F0001);    
    	 if(mode == USB2x) usb2.sendData((unsigned int)0x000F0001);
     } else { //disable USB_SYNC
  	 usb.sendData((unsigned int)0x000F0000);
     	 if(mode == USB2x) usb2.sendData((unsigned int)0x000F0000);
      
     }
     closeUSBHandles();
}
\end{lstlisting}



\subsubsection{Destination}

%\textbf{ACDC-Firmware:SRC/dff\_async\_rst.vhd}  % project:path/to/filename.txt
\begin{lstlisting}[style=vhdl]
	
\end{lstlisting}



\subsection{0xD - Align LVDS}\label{inst:alignlvds}

This command aligns the LVDS system between the central card and any acdc boards. The LVDS system is the RJ-45 connection.

\subsubsection{Bit Fields}

\begin{figure}[hb!]
\begin{center}
\begin{bytefield}[endianness=big,bitwidth=1.1em]{32}

	\bitheader{0-31} \\ 
	\bitbox{12}{\color{lightgray}\rule{\width}{\height}} & \bitbox{4}{0xD} & \bitbox{16}{\color{lightgray}\rule{\width}{\height}} \\

\end{bytefield}
\caption{Command 0xD bit fields}
\end{center}
\end{figure}

\begin{table}[h]
\begin{tabular}{p{0.06\textwidth}p{0.30\textwidth}p{0.64\textwidth}}
	Bits & Name &  Description\\
	\hline
	19-16 & \textbf{0xD} & Command marker \\
	\hline
\end{tabular}
\end{table}



\subsubsection{Source}

\textbf{acdc-daq:src/DAQinstruction.cpp}
\begin{lstlisting}[style=c++]
void SuMo::align_lvds()
{
     createUSBHandles();
     usb.sendData((unsigned int)0x000D0000);  //toggle align process
     if(mode == USB2x) usb2.sendData((unsigned int)0x000D0000);
     closeUSBHandles();
}
\end{lstlisting}



\subsubsection{Destination}

%\textbf{ACDC-Firmware:SRC/dff\_async\_rst.vhd}  % project:path/to/filename.txt
\begin{lstlisting}[style=vhdl]
	
\end{lstlisting}


\subsection{0xA - Toggle LED}\label{inst:toggleled}

This command toggles the LED on all connected boards.

\subsubsection{Bit Fields}

\begin{figure}[hb!]
\begin{center}
\definecolor{lightgray}{gray}{0.8}
\begin{bytefield}[endianness=big,bitwidth=1.1em]{32}

	\bitheader{0-31} \\ 
	\bitbox{3}{\color{lightgray}\rule{\width}{\height}} & \bitbox{4}{brd adr} & 
	\bitbox{5}{\color{lightgray}\rule{\width}{\height}} & \bitbox{4}{0xA} & 
	\bitbox{15}{\color{lightgray}\rule{\width}{\height}} & \bitbox{1}{\rotatebox{90}{\small En}}\\

\end{bytefield}
\caption{Command 0xA bit fields}
\end{center}
\end{figure}

\begin{table}[h]
\begin{tabular}{p{0.06\textwidth}p{0.30\textwidth}p{0.64\textwidth}}
	Bits  & Name          & Description                              \\ \hline
	28-25 & board address & The address of the board. Default is 0xF \\
	19-16 & \textbf{0xA}  & Command marker                           \\
	0     & enable        & Enable the leds                          \\ \hline
\end{tabular}
\end{table}



\subsubsection{Source}

\textbf{acdc-daq:src/DAQinstruction.cpp}
\begin{lstlisting}[style=c++]
void SuMo::toggle_LED(bool EN)
{
     unsigned int boardAdr_all = 15;

     createUSBHandles();
     unsigned int send_word = 0x000A0000;
     send_word = send_word | boardAdr_all << boardAdrOffset;

     if(EN != false){
       usb.sendData(send_word | 0x1);
       if(mode == USB2x) usb2.sendData(send_word | 0x1);
     } 
     else{
       usb.sendData(send_word);
       if(mode == USB2x) usb2.sendData(send_word);
     }
     closeUSBHandles();
}
\end{lstlisting}



\subsubsection{Destination}

%\textbf{ACDC-Firmware:SRC/dff\_async\_rst.vhd}  % project:path/to/filename.txt
\begin{lstlisting}[style=vhdl]
	
\end{lstlisting}

\newpage


\subsection{0x7 - Set Self-trigger Lo}\label{inst:selftriggerlo}

This command sends certain trigger commands

\subsubsection{Bit Fields}

\begin{figure}[hb!]
\begin{center}
\begin{bytefield}[endianness=big,bitwidth=1.1em]{32}

	\bitheader{0-31} \\ 
	\bitbox{3}{\color{lightgray}\rule{\width}{\height}} & \bitbox{4}{brd adr} & 
	\bitbox{5}{\color{lightgray}\rule{\width}{\height}} & \bitbox{4}{0x7} & 
	\bitbox{4}{0x0} & \bitbox{1}{0} & 
	\bitbox{4}{coinc} & \bitbox{1}{\rotatebox{90}{\tiny val}} &
	\bitbox{1}{\rotatebox{90}{\tiny coin}} & \bitbox{1}{\rotatebox{90}{\tiny sma}} &
	\bitbox{1}{\rotatebox{90}{\tiny sign}} & \bitbox{1}{\rotatebox{90}{\tiny rate}} &
	\bitbox{1}{\rotatebox{90}{\tiny sys}} & \bitbox{1}{\rotatebox{90}{\tiny en}}\\

\end{bytefield}
\caption{Command 0x7 bit fields}
\end{center}
\end{figure}

\begin{table}[h]
\begin{tabular}{p{0.06\textwidth}p{0.30\textwidth}p{0.64\textwidth}}
	Bits  & Name               & Description                                                    \\ \hline
	28-25 & board address      & The address of the board. Default is 0xF                       \\
	19-16 & \textbf{0x7}       & Command marker                                                 \\
	15-12 & \textbf{0x0}       & Optional command marker                                        \\
	10-7  & coincidence window & The window of unknown units for the coincidence (\textless 15) \\
	6     & use trig valid     & Use trig valid as a reset on AC/DC                             \\
	5     & use coincidence    & Use channel coincidence                                        \\
	4     & use board sma trig & Use SMA input on AC/DC board for trigger                       \\
	3     & trig sign          & 1 for rising edge, 0 for falling edge                          \\
	2     & rate only          &                                                                \\
	1     & sys trig option    &                                                                \\
	0     & trig enable        & Enables self-trigger                                           \\ \hline
\end{tabular}
\end{table}



\subsubsection{Source}

\textbf{acdc-daq:src/DAQinstruction.cpp}
\begin{lstlisting}[style=c++]
void SuMo::set_self_trigger_lo(     bool ENABLE_TRIG,
				    bool SYS_TRIG_OPTION,
				    bool RATE_ONLY,
				    bool TRIG_SIGN,
				    bool USE_BOARD_SMA_TRIG,
				    bool USE_COINCIDENCE,
				    bool USE_TRIG_VALID_AS_RESET,
				    unsigned int coinc_window,
				    unsigned int boardAdr,
				    int device)
{
    const unsigned int hi_cmd = 0x00070000;
    unsigned int send_word = hi_cmd | 0 << 11
      | USE_TRIG_VALID_AS_RESET << 6
      | USE_COINCIDENCE << 5
      | USE_BOARD_SMA_TRIG << 4
      | TRIG_SIGN << 3 | RATE_ONLY << 2
      | SYS_TRIG_OPTION << 1 | ENABLE_TRIG
      | coinc_window << 7
      | boardAdr << boardAdrOffset;
    //printf("%i\n", send_word);

    createUSBHandles();

    if(device == 0)                  usb.sendData((unsigned int)send_word);
    if(device == 1 && mode == USB2x) usb2.sendData((unsigned int)send_word);

    closeUSBHandles();
}
\end{lstlisting}



\subsubsection{Destination}

%\textbf{ACDC-Firmware:SRC/dff\_async\_rst.vhd}  % project:path/to/filename.txt
\begin{lstlisting}[style=vhdl]
	
\end{lstlisting}



\subsection{0x7 - Set Self-trigger hi}\label{inst:selftriggerhi}

This command sends certain trigger commands

\subsubsection{Bit Fields}

\begin{figure}[hb!]
\begin{center}
\definecolor{lightgray}{gray}{0.8}
\begin{bytefield}[endianness=big,bitwidth=1.1em]{32}

	\bitheader{0-31} \\ 
	\bitbox{3}{\color{lightgray}\rule{\width}{\height}} & \bitbox{4}{brd adr} & 
	\bitbox{5}{\color{lightgray}\rule{\width}{\height}} & \bitbox{4}{0x7} & 
	\bitbox{4}{0x8} & \bitbox{1}{1} &
	\bitbox{5}{chan} & \bitbox{3}{asic} & \bitbox{3}{width}

\end{bytefield}
\caption{Command 0x7 bit fields}
\end{center}
\end{figure}

\begin{table}[h]
\begin{tabular}{p{0.06\textwidth}p{0.30\textwidth}p{0.64\textwidth}}
	Bits  & Name                    & Description                                                     \\ \hline
	28-25 & board address           & The address of the board.                                       \\
	19-16 & \textbf{0x7}            & Command marker                                                  \\
	15-12 & \textbf{0x8}            & Optional command marker                                         \\
	10-6  & channel coincidence min & Number of coincident channels to enable trigger (\textless 30)  \\
	5-3   & asic coincidence min    & Number of coincident asic chips to enable trigger (\textless 5) \\
	2-0   & coincidence pulse width & Width of coincidence pulse in unknown units (\textless 7)       \\ \hline
\end{tabular}
\end{table}



\subsubsection{Source}

\textbf{acdc-daq:src/DAQinstruction.cpp}
\begin{lstlisting}[style=c++]
void SuMo::set_self_trigger_hi(unsigned int coinc_pulse_width,
	unsigned int asic_coincidence_min,
	unsigned int channel_coincidence_min,
	unsigned int boardAdr,
	int device)
	{
	const unsigned int hi_cmd = 0x00078000;
	unsigned int send_word = hi_cmd | 1 << 11
	| channel_coincidence_min << 6
	| asic_coincidence_min << 3
	| coinc_pulse_width
	| boardAdr << boardAdrOffset;
	//printf("%x\n", send_word);

	createUSBHandles();

	if(device == 0) usb.sendData((unsigned int)send_word);
	if(device == 1 && mode == USB2x) usb2.sendData((unsigned int)send_word);

	closeUSBHandles();
}
\end{lstlisting}



\subsubsection{Destination}

%\textbf{ACDC-Firmware:SRC/dff\_async\_rst.vhd}  % project:path/to/filename.txt
\begin{lstlisting}[style=vhdl]
	
\end{lstlisting}

\newpage


\subsection{0xA - Read ACDC Ram}\label{inst:read_acdcRam}

% Add description of command here
\color{red}{ \LARGE{ THIS COMMAND DOES NOT FOLLOW ERIC'S BASIC PROTOCOL OUTLINE (\autoref{sec:intro}, \autoref{byte:protocol}) }}

\color{black}

\subsubsection{Bit Fields}
\begin{figure}[hb!]
\begin{center}
	\begin{bytefield}[endianness=big,bitwidth=1.1em]{32}
	\bitheader{0-31} \\ 
	% Fill in bytefield here
	\bitbox{3}{\color{lightgray}\rule{\width}{\height}} & \bitbox{4}{brd adr} &
	\bitbox{5}{\color{lightgray}\rule{\width}{\height}} & \bitbox{4}{0xA} &
	\bitbox{12}{\color{lightgray}\rule{\width}{\height}} & \bitbox{4}{0x6} \\ 
	\end{bytefield}
\caption{Command 0xA bit fields} % Appropriate caption here
\end{center}
\end{figure}

\begin{table}[h]	
% fill out bit information here
\begin{tabular}{p{0.06\textwidth}p{0.30\textwidth}p{0.64\textwidth}}
	Bits  & Name          & Description                             \\ \hline
	28-25 & board address & The address of the board (default = 15) \\
	19-16 & \textbf{0xA}  & Command marker                          \\
	 3-0  & \textbf{0x6}  & Optional command marker                 \\ \hline
\end{tabular}
\end{table}



\subsubsection{Source}
\textbf{acdc-daq:src/DAQinstruction.cpp}
% Put c++ source code here
\begin{lstlisting}[style=c++]
void SuMo::readACDC_RAM(int device, unsigned int boardAdr)
{
	unsigned int boardAdr_override  = 15;
	
	createUSBHandles();
	unsigned int send_word = 0x000A0006;
	send_word = send_word | boardAdr << boardAdrOffset;
	
	if(device == 0)                 usb.sendData(send_word);
	if(device == 1 && mode==USB2x)  usb2.sendData(send_word);
	
	closeUSBHandles();
}
\end{lstlisting}



\subsubsection{Destination}
\textbf{project:path/to/filename.vhd}  % project:path/to/filename.txt
%put vhdl firmware code here
\begin{lstlisting}[style=vhdl]

\end{lstlisting}



\subsection{0x2 -Toggle Cal}\label{inst:togglecal}

% Add description of command here

\subsubsection{Bit Fields}
\begin{figure}[hb!]
\begin{center}
	\begin{bytefield}[endianness=big,bitwidth=1.1em]{32}
	\bitheader{0-31} \\ 
	% Fill in bytefield here
	\bitbox{3}{\color{lightgray}\rule{\width}{\height}} & \bitbox{4}{brd adr} &
	\bitbox{5}{\color{lightgray}\rule{\width}{\height}} & \bitbox{4}{0x2} &
	\bitbox{16}{channels} \\
	\end{bytefield}
\caption{0x2 - Cal enabled} % Appropriate caption here
\end{center}
\end{figure}

\begin{figure}[hb!]
	\begin{center}
		\begin{bytefield}[endianness=big,bitwidth=1.1em]{32}
			\bitheader{0-31} \\ 
			% Fill in bytefield here
			\bitbox{3}{\color{lightgray}\rule{\width}{\height}} & \bitbox{4}{brd adr} &
			\bitbox{5}{\color{lightgray}\rule{\width}{\height}} & \bitbox{4}{0x2} &
			\bitbox{16}{\color{lightgray}\rule{\width}{\height}}  \\ 
		\end{bytefield}
		\caption{0x2 - Cal disabled} % Appropriate caption here
	\end{center}
\end{figure}


\begin{table}[h]	
% fill out bit information here
\begin{tabular}{p{0.06\textwidth}p{0.30\textwidth}p{0.64\textwidth}}
	Bits  & Name          & Description                             \\ \hline
	28-25 & board address & The address of the board (default = 15) \\
	19-16 & \textbf{0x2}  & Command marker                          \\
	15-0  & channels      & Channels (default = 0x7FFF)             \\
	      &               & Not used if cal is disabled             \\ \hline
\end{tabular}
\end{table}



\subsubsection{Source}
\textbf{acdc-daq:src/DAQinstruction.cpp}
% Put c++ source code here
\begin{lstlisting}[style=c++]
void SuMo::toggle_CAL(bool EN,  int device)
{
	createUSBHandles();
	
	unsigned int send_word = 0x00020000;
	unsigned int channels  = 0x7FFF;
	unsigned int boardAdr  = 15;
	
	if(EN){
		send_word = send_word | boardAdr << boardAdrOffset | channels;
		if(device == 0)                usb.sendData(send_word);
		if(device == 1 && mode==USB2x) usb2.sendData(send_word);
	} else{
		send_word = send_word | boardAdr << boardAdrOffset;
		if(device == 0)                usb.sendData(send_word);
		if(device == 1 && mode==USB2x) usb2.sendData(send_word);
	}
closeUSBHandles();
}
\end{lstlisting}



\subsubsection{Destination}
\textbf{project:path/to/filename.vhd}  % project:path/to/filename.txt
%put vhdl firmware code here
\begin{lstlisting}[style=vhdl]

\end{lstlisting}



\subsection{0x3 -Set Pedestal Value}\label{inst:setpedvalue}

% Add description of command here

\subsubsection{Bit Fields}
\begin{figure}[hb!]
\begin{center}
	\begin{bytefield}[endianness=big,bitwidth=1.1em]{32}
	\bitheader{0-31} \\ 
	% Fill in bytefield here
	\bitbox{3}{\color{lightgray}\rule{\width}{\height}} & \bitbox{4}{brd adr} &
	\bitbox{5}{psec mask} & \bitbox{4}{0x3} &
	\bitbox{4}{\color{lightgray}\rule{\width}{\height}} & \bitbox{12}{value} \\ 
	\end{bytefield}
\caption{Command 0x3 bit fields} % Appropriate caption here
\end{center}
\end{figure}

\begin{table}[h]	
% fill out bit information here
\begin{tabular}{p{0.06\textwidth}p{0.30\textwidth}p{0.64\textwidth}}
	Bits  & Name          & Description                             \\ \hline
	28-25 & board address & The address of the board (default = 15) \\
	24-20 & psec mask     & Mask for chip addresses                 \\
	19-16 & \textbf{0x2}  & Command marker                          \\
	12-0  & value         & Pedestal value (default = 0x800)        \\ \hline
\end{tabular}
\end{table}



\subsubsection{Source}
\textbf{acdc-daq:src/DAQinstruction.cpp}
% Put c++ source code here
\begin{lstlisting}[style=c++]
void SuMo::set_pedestal_value(  unsigned int PED_VALUE,
unsigned int boardAdr,
int device,
unsigned int psec_mask)
{
	createUSBHandles();
	const unsigned int hi_cmd = 0x00030000;
	unsigned int send_word = hi_cmd | PED_VALUE
	| boardAdr << boardAdrOffset
	| psec_mask << psecAdrOffset;
	
	if(device == 0)                 usb.sendData(send_word);
	if(device == 1 && mode==USB2x)  usb2.sendData(send_word);
	
	closeUSBHandles();
}
\end{lstlisting}



\subsubsection{Destination}
\textbf{project:path/to/filename.vhd}  % project:path/to/filename.txt
%put vhdl firmware code here
\begin{lstlisting}[style=vhdl]

\end{lstlisting}

\newpage


%\bibliographystyle{plain-annote}
%\bibliography{mybibliography}


\end{document}
\end

